\documentclass[pdf]{beamer}
\mode<presentation>{\usetheme{Warsaw}}

% packages
\usepackage{listings}
% \usepackage{upquote}

% preamble
\title{Introduction to Lisp Programming with Clojure}
\author{Wojciech Gac}

\begin{document}

\begin{frame}
  \titlepage
\end{frame}

\begin{frame}{Outline}
  \pause
  \begin{itemize}
  \item What is Lisp?
    \pause
  \item History
    \pause
  \item Clojure Basics
    \pause
  \item Tooling
    \pause
  \item Advanced Topics
    \pause
  \item Where to Go Next?
  \end{itemize}
\end{frame}

% What is Lisp?
\begin{frame}{What is Lisp?}
  \pause
  \begin{itemize}
  \item Grew out of a purely mathematical idea of John McCarthy
    \pause
  \item Another way to describe universal computation (equivalent to Turing Machines, Lambda Calculus, Cellular Automata, etc.)
    \pause
  \item Syntax \& Data are the same (homoiconicity)
    \pause
  \item Programs typically run as live memory images
    \pause
  \item Macros allow code generation (programs writing programs, writing programs...)
    \pause
  \item Supports many programming paradigms (OOP, Aspect, Logical, Imperative, Symbolic), but is fundamentally functional
  \end{itemize}
\end{frame}

\begin{frame}{What is Lisp?}
  Types of Lisp:
  \pause
  \begin{itemize}
  \item Lisp-1 (single namespace)
    \pause
  \item Lisp-2 (separate namespaces for variables and functions)
    \pause
  \item Lisp-like languages
  \end{itemize}
\end{frame}

\begin{frame}{What is Lisp?}
  Lisp-1 examples:
  \pause
  \begin{itemize}
  \item Scheme
    \pause
  \item Clojure
    \pause
  \item Arc
    \pause
  \item PicoLisp
  \end{itemize}
\end{frame}

\begin{frame}{What is Lisp?}
  Lisp-2 examples:
  \pause
  \begin{itemize}
  \item Common Lisp
    \pause
  \item Emacs Lisp
    \pause
  \item AutoLisp
  \end{itemize}
\end{frame}

\begin{frame}{What is Lisp?}
  Lisp-like examples:
  \pause
  \begin{itemize}
  \item LFE (Lisp-Flavored Erlang)
    \pause
  \item Nu (Scripting language for OS X)
    \pause
  \item Hy (Lisp-like Python)
    \pause
  \item Liskell (Lisp syntax for Haskell)
    \pause
  \end{itemize}
\end{frame}

\begin{frame}{What is Lisp?}
  Scheme:
  \pause
  \begin{itemize}
  \item Lisp-1 (shared namespace for variables and functions)
    \pause
  \item Proved that lexical scope can be efficient
    \pause
  \item Pure functional style preferred, but not enforced
    \pause
  \item Support for continuations
    \pause
  \item Some better-known implementations:
    \pause
    \begin{itemize}
    \item GNU Guile
      \pause
    \item Racket
      \pause
    \item Chez Scheme
      \pause
    \item Bigloo
      \pause
    \item Chicken Scheme
    \end{itemize}
  \end{itemize}
\end{frame}

\begin{frame}{What is Lisp?}
  Common Lisp:
  \pause
  \begin{itemize}
  \item Lisp-2 (separate namespaces for variables and functions)
    \pause
  \item Mature, industrial-strength dialect designed in 1980s
    \pause
  \item CLOS - powerful object system based on generic functions
    \pause
  \item Meta-Object Protocol - entrypoint for full CLOS customization
    \pause
  \item Not too restrictive about side-effects
    \pause
  \item Some implementations:
    \pause
    \begin{itemize}
    \item Allegro Common Lisp (commercial)
      \pause
    \item LispWorks (commercial)
      \pause
    \item SBCL
      \pause
    \item CLISP
      \pause
    \item Clozure CL
      \pause
    \item Armed Bear Common Lisp (runs on JVM)
      \pause
    \item Movitz (defunct bare-metal subset of CL)
    \end{itemize}
  \end{itemize}
\end{frame}

\begin{frame}{What is Lisp?}
  Clojure:
  \pause
  \begin{itemize}
  \item Lisp-1
    \pause
  \item Runs on JVM (also on .NET CLR)
    \pause
  \item Strong Java interoperability layer
    \pause
  \item Seamless integration with Java code (via JAR)
  \end{itemize}
\end{frame}

% History
\begin{frame}{History}
  Timeline:
  \pause
  \begin{itemize}
  \item 1958 - John McCarthy published the original paper
    \pause
  \item 1962 - First Lisp compiler written at MIT
    \pause
  \item 1965 - Maclisp created at MIT AI Lab
    \pause
  \item 1968 - Interlisp created at BBN/Xerox
    \pause
  \item 1973 - First Lisp Machine created
    \pause
  \item 1975 - Scheme invented by Steele \& Sussman
    \pause
  \item 1980 - Symbolics \& LMI market Lisp Machines
    \pause
  \item 1984 - Common Lisp first appeared
    \pause
  \item 1985 - Emacs Lisp at the core of GNU Emacs
    \pause
  \item 2007 - Rich Hickey created Clojure
  \end{itemize}
\end{frame}

% Clojure Basics
\begin{frame}{Clojure Basics}
  Data types:
  \pause
  \begin{itemize}
  \item Numbers (integers, reals, bigint, bigdec, rationals)
    \pause
  \item Chars e.g. \lstinline{\\a, \\?}
    \pause
  \item Strings
    \pause
  \item Symbols e.g. \lstinline{test, lol, hi-there}
    \pause
  \item Boolean values: \lstinline{true, false}
    \pause
  \item Non-value: \lstinline{nil}
  \end{itemize}
\end{frame}

% Tooling
\begin{frame}{Tooling}
  \begin{itemize}
    \pause
  \item Leiningen (\textit{leiningen.org})
    \pause
  \item GNU Emacs/CIDER (\textit{cider.readthedocs.io})
    \pause
  \item IntelliJ/Cursive (\textit{cursive-ide.com})
  \end{itemize}
\end{frame}

\begin{frame}[fragile]{Leiningen Walkthrough}
  \begin{itemize}
    \pause
  \item Create new project
\begin{verbatim}
$ lein new app my-tutorial-app
\end{verbatim}
  \end{itemize}
  
\end{frame}

% Examples

\begin{frame}[fragile]{Examples}
  Tail recursion (JVM does not support tail-call optimization):
  \pause
\begin{verbatim}
(defn factorial
  "Tail-recursive implementation of factorial"
  [n]
  (loop [n n
         acc 1]
    (if (zero? n)
      acc
      (recur (dec n) (*' n acc)))))
\end{verbatim}
\end{frame}

\begin{frame}[fragile]{Examples}
  Mapping anonymous function to two sequences and reducing the result with addition:
  \pause
\begin{verbatim}
(let [a (range 1 6)
      b (range 13 18)]
  (reduce + 1000000 (map #(+ (* 1000 %1) (* 10 %2)) a b)))
\end{verbatim}
  \pause
\begin{verbatim}
=> 1015750
\end{verbatim}
\end{frame}

\begin{frame}[fragile]{Examples}
  Threading operator:
  \pause
\begin{verbatim}
(->>
 (range 10)
 (map inc)
 (partition 3)
 (map #(reduce + %)))
\end{verbatim}
  \pause
\begin{verbatim}
=> (6 15 24)
\end{verbatim}
\end{frame}

\begin{frame}[fragile]{Examples}
  Matrix transposition:
  \pause
\begin{verbatim}
(defn transpose
  "Transpose matrix by selecting corresponding elements from rows"
  [matrix]
  (apply map list matrix))
\end{verbatim}
  \pause
\begin{verbatim}
=> ((1 4 7) (2 5 8) (3 6 9))
\end{verbatim}
\end{frame}

% Advanced Topics
\begin{frame}{Advanced Topics}
  \begin{itemize}
  \item Macros
    \pause
  \item core.async
    \pause
  \item Integrating Clojure with Java
  \end{itemize}
\end{frame}

\begin{frame}[fragile]{Macros}
  Anaphoric macros - contain local bindings for later use inside the macro body\\
  \pause
  Anaphoric if:
  \pause
\begin{verbatim}
(defmacro aif
  [condition-var condition when-true when-false]
  `(let [~condition-var ~condition]
     (if ~condition-var ~when-true ~when-false)))
\end{verbatim}
\end{frame}

\begin{frame}[fragile]{Macros}
  Anaphoric when:
  \pause
\begin{verbatim}
(defmacro awhen
  [condition-var condition & body]
  `(let [~condition-var ~condition]
     (when ~condition-var ~@body)))
\end{verbatim}
\end{frame}

\begin{frame}[fragile]{Macros}
  Verbose variable definition:
  \pause
\begin{verbatim}
(defmacro verbose-def
  [var val]
  (println "Defining variable " var " with value " val)
  `(def ~var ~val))
\end{verbatim}
\end{frame}

% Where to Go Next?
\begin{frame}{Where to Go Next?}
  \pause
  \begin{itemize}
  \item Books:
    \pause
    \begin{itemize}
    \item \textit{Clojure for the Brave and True} \\ - {\scriptsize \textsc{braveclojure.com}}
      \pause
    \item \textit{Practical Clojure} \\ - {\scriptsize \textsc{apress.com/us/book/9781430272311}}
      \pause
    \item \textit{Living Clojure} \\ - {\scriptsize \textsc{shop.oreilly.com/product/0636920034292.do}}
      \pause
    \item \textit{Structure and Interpretation of Computer Programs/SICP} \\ - {\scriptsize \textsc{sicpebook.wordpress.com/ebook/}}
    \end{itemize}
    \pause
  \item Tutorials:
    \pause
    \begin{itemize}
    \item \textit{Introduction to Clojure} \\ - {\scriptsize \textsc{clojure-doc.org/articles/tutorials/introduction.html}}
      \pause
    \item \textit{Clojure Koans} \\ - {\scriptsize \textsc{clojurekoans.com}}
    \end{itemize}
  \end{itemize}
\end{frame}


% Test
% \begin{frame}{Timing example}
%   \begin{table}[bt]
%     \begin{tabular}{|c|c|c|}
%     % \centering
%     \hline
%   \uncover<3-5,7-9>{1} & \uncover<4-6,8-10>{2} & \uncover<5-7,9-11>{3} \\
%     \hline
%   \uncover<2-4,6-8>{4} & \uncover<9-11,13-15>{5} & \uncover<6-8,10-12>{6} \\
%     \hline
%   \uncover<1-3,5-7>{7} & \uncover<8-10,12-14>{8} & \uncover<7-9,11-13>{9} \\
%     \hline
%     \end{tabular}
%   \end{table}
% \end{frame}

\end{document}