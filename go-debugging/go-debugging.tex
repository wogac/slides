\documentclass[pdf,xcolor=dvipsnames,noparindent]{beamer}
\mode<presentation>{\usetheme{Warsaw}}

% packages
\usepackage{listings}
% \usepackage{upquote}

\lstset{numbers=left, numberstyle=\footnotesize, stepnumber=1,firstnumber=1,
    numbersep=5pt,
    stringstyle=5pt,
    basicstyle=\footnotesize,
    keepspaces=true, tabsize=4,
    showstringspaces=false
    % backgroundcolor=\color{SpringGreen}
}

% preamble
\title{Debugging Go inside Docker containers}
\author{Wojciech Gac}

\begin{document}

\begin{frame}
  \titlepage
\end{frame}

\begin{frame}{Outline}
  \pause
  \begin{itemize}
  \item The Problem
    \pause
  \item Delve Debugger
    \pause
  \item Container Preparation
    \pause
  \item Debugging
    \pause
  \item Further Reading
  \end{itemize}
\end{frame}

\begin{frame}
  \frametitle{The Problem}
  We'd like to be able to do the following:
  \pause
  \begin{itemize}
  \item Using a local instance of VSCode...
    \pause
  \item ...and a dockerized instance of a Go application...
    \pause
  \item ...connect to a running process \emph{inside} the container...
    \pause
  \item ...interrupt execution with breakpoints and modify variables on the fly
  \end{itemize}
  
\end{frame}

\begin{frame}
  \frametitle{Delve Debugger}
  \textbf{Delve} (dlv) is the primary debugger for Go, similar in
  spirit to \textbf{GDB}. It has the following main features:
  \pause
  \begin{itemize}
  \item Setting and managing of breakpoints and checkpoints
    \pause
  \item Code walking - stepping into and over functions, instruction-level stepping
    \pause
  \item Goroutine and thread debugging
    \pause
  \item Tracing
    \pause
  \item Type introspection
    \pause
  \item ...and more (but no function calls yet)
  \end{itemize}
  
\end{frame}

\begin{frame}[fragile]
  \frametitle{Delve Debugger}
  Standalone usage basics:
  \pause
  \begin{itemize}
  \item Installation
\begin{verbatim}
$ go get -u github.com/go-delve/delve/cmd/dlv
\end{verbatim}
    \pause
  \item Debugging binaries (starting at \emph{main()})
\begin{verbatim}
$ dlv debug path/to/binary
\end{verbatim}
    \pause
  \item Debugging package tests
\begin{verbatim}
$ dlv test path/to/directory/containing/tests
\end{verbatim}
    \pause
  \end{itemize}
  Once dropped into Delve's command line, you can set breakpoints, inspect source code and eventually run \emph{continue}
\end{frame}

\begin{frame}[fragile]
  \frametitle{Container Preparation}
  Sample application (file: \emph{tutorial/tutorial.go}):
  \pause
\begin{lstlisting}
package main

import (
	"fmt"
	"log"
	"net/http"
)

func main() {
	http.HandleFunc("/", func(w http.ResponseWriter,
                             r *http.Request) {
		nr := 1
		msg := `Hello World`
		w.Write([]byte(fmt.Sprintf(
                        "Message: \"%s\" Number: %d", 
                        msg, nr)))
	})
	log.Fatal(http.ListenAndServe(":8080", nil))
}
\end{lstlisting}
\end{frame}

\begin{frame}[fragile]
  \frametitle{Container Preparation}
  Dockerfile:
\begin{lstlisting}[]
FROM golang:1.11.10-alpine3.8
ENV CGO_ENABLED 0
ENV GOPATH /opt/go:$GOPATH
ENV PATH /opt/go/bin:$PATH
ADD . /opt/go/src/tutorial
WORKDIR /opt/go/src/tutorial
 
RUN apk add --no-cache git
RUN go get github.com/derekparker/delve/cmd/dlv
RUN go build -o tutorial tutorial.go
CMD ["./tutorial"]
\end{lstlisting}
\end{frame}

\begin{frame}[fragile]
  \frametitle{Container Preparation}
  docker-compose.yaml
\begin{lstlisting}
version: '2'
services:
  tutorial:
    build: .
    security_opt:
      - seccomp:unconfined
    entrypoint: >
          dlv debug tutorial -l :40000 
          --headless=true --api-version=2 --log=true
    ports:
      - "8080:8080"
      - "40000:40000"
    expose:
      - "8080"
      - "40000"
\end{lstlisting}
\end{frame}

\begin{frame}[fragile]
  \frametitle{Container Preparation}
  VSCode configuration (file: \emph{launch.json}):
\begin{lstlisting}[]
{
    "version": "0.2.0",
    "configurations": [
        {
            "name": "Remote Docker",
            "type": "go",
            "request": "launch",
            "mode": "remote",
            "remotePath": "/opt/go/src/tutorial",
            "port": 40000,
            "host": "172.19.0.2",
            "program": "${workspaceRoot}",
            "env": {},
            "args": []
        }
    ]
}
\end{lstlisting}
\end{frame}

\begin{frame}
  \frametitle{Connecting}
  Once the application has been launched (through
  \emph{docker-compose}), there is a debugging server running inside
  the container, exposed at the port chosen by the user (in our case
  \emph{40000}).
  \newline
  \newline
  \pause
  Now there are two main scenarios of what can be done next:
  \pause
  \begin{itemize}
  \item Connect through the CLI client
    \pause
  \item Connect from within VSCode
  \end{itemize}
\end{frame}

\begin{frame}[fragile,fragile,fragile]
  \frametitle{Connecting from CLI}
  In order to connect using Delve's own client:
\begin{verbatim}
$ dlv connect --api-version 2 172.19.0.2:40000
\end{verbatim}
  \pause This will drop you into the debugger, directly connected to
  the containerized process.
  \pause
  \newline
  Due to source-relative offsets involved in debugging it might be easier to run \emph{dlv} from within the container in question:
\begin{verbatim}
$ dlv connect --api-version 2 :40000
\end{verbatim}
\end{frame}

\begin{frame}
  \frametitle{Connecting from VSCode}
  With the above configuration (\emph{launch.json}) you should be able
  to directly connect to the debugging API.
  \pause
  \newline
  \newline
  Steps:
  \pause
  \begin{itemize}
  \item Go to the \emph{Debug} window (\emph{Ctrl+Shift+D})
    \pause
  \item From the \emph{Debug} list select your config (in our case: \emph{Remote Docker})
    \pause
  \item After setting breakpoints and pressing \emph{Start Debugging} (or \emph{F5}) the debugging session will proceed
  \end{itemize}
\end{frame}

\begin{frame}
  \frametitle{Further Reading}
  Further reading:
  \begin{itemize}
  \item \emph{https://github.com/go-delve/delve} - official Delve documentation
  \item \emph{https://www.jamessturtevant.com/posts/Using-the-Go-Delve-Debugger-from-the-command-line/} - article on CLI debugging with Delve
  \item \emph{https://github.com/Microsoft/vscode-go/wiki/Debugging-Go-code-using-VS-Code} - official guide to debugging with Delve and VSCode
  \item \emph{https://blog.jetbrains.com/go/2018/04/30/debugging-containerized-go-applications/} - same problem with GoLand
  \end{itemize}
\end{frame}

% Test
% \begin{frame}{Timing example}
%   \begin{table}[bt]
%     \begin{tabular}{|c|c|c|}
%     % \centering
%     \hline
%   \uncover<3-5,7-9>{1} & \uncover<4-6,8-10>{2} & \uncover<5-7,9-11>{3} \\
%     \hline
%   \uncover<2-4,6-8>{4} & \uncover<9-11,13-15>{5} & \uncover<6-8,10-12>{6} \\
%     \hline
%   \uncover<1-3,5-7>{7} & \uncover<8-10,12-14>{8} & \uncover<7-9,11-13>{9} \\
%     \hline
%     \end{tabular}
%   \end{table}
% \end{frame}

\end{document}